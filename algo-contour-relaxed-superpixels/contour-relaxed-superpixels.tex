\documentclass[12pt,a4paper]{article}

\usepackage{amsmath}
\usepackage[font=footnotesize]{caption}
\usepackage[section]{algorithm}
\captionsetup[algorithm]{font=footnotesize}
\usepackage[numbered]{algo}

\begin{document}

	\begin{algorithm}[t]
		\begin{algo}{CRS}{\label{algo:related-work-contour-relaxed}\qinput{color image $I$, number of superpixels $K$}\qoutput{superpixel segmentation $S$}}
			\qcom{The step size $R$ can be derived from the image size $W \times H$ and $K$:}\\
			initialize $S$ as regular grid with step size $R$\\
			initialize $\boldsymbol \theta$ using sufficient statistics (e.g. Gaussian)\\
			\qfor $t = 1$ \qto $T$\\
				\qcom{Originally, the image is traversed multiple times using different directions to avoid a directional bias \cite{ConradMertzMester}:}\\
				\qfor $n = 1$ \qto $N$\\
				\qif $x_n$ is a boundary pixel\\
					\qcom{This can be evaluated by taking $\boldsymbol \theta$ as constant; Conrad et al. suggest to minimize the negative logarithm of \eqref{eq:related-work-contour-relaxed-maximize} instead:}\\
					\qthen assign $x_n$ to the label maximizing \eqref{eq:related-work-contour-relaxed-maximize}\qfi\qrof\qrof\\
			\qreturn $S$
		\end{algo}
		\caption{The algorithm to maximize the energy given in equation \eqref{eq:related-work-contour-relaxed-energy} to obtain Contour Relaxed Superpixels \cite{ConradMertzMester}.}
		\label{fig:related-work-contour-relaxed-algorithm}
	\end{algorithm}
	
	\begin{align}
		\label{eq:related-work-contour-relaxed-maximize}
		E(S) = \kappa \kappa' \exp(- N_{e} C_{e} - N_{v} C_{v}) \prod_{S_i} \prod_{x_m \in S_i} \prod_{c=1}^3 p(I_c(x_m) | \theta_{i,c})
	\end{align}
	
	\begin{thebibliography}{1}
		\bibitem{ConradMertzMester}
		C. Conrad, M. Mertz, R. Mester.
		\emph{Contour-relaxed superpixels}.
		Energy Minimization Methods in Computer Vision and Pattern Recognition, 2013.
	\end{thebibliography}

\end{document}